%% The MIT License (MIT)
%%
%% Copyright (c) 2015 Daniil Belyakov
%%
%% Permission is hereby granted, free of charge, to any person obtaining a copy
%% of this software and associated documentation files (the "Software"), to deal
%% in the Software without restriction, including without limitation the rights
%% to use, copy, modify, merge, publish, distribute, sublicense, and/or sell
%% copies of the Software, and to permit persons to whom the Software is
%% furnished to do so, subject to the following conditions:
%%
%% The above copyright notice and this permission notice shall be included in all
%% copies or substantial portions of the Software.
%%
%% THE SOFTWARE IS PROVIDED "AS IS", WITHOUT WARRANTY OF ANY KIND, EXPRESS OR
%% IMPLIED, INCLUDING BUT NOT LIMITED TO THE WARRANTIES OF MERCHANTABILITY,
%% FITNESS FOR A PARTICULAR PURPOSE AND NONINFRINGEMENT. IN NO EVENT SHALL THE
%% AUTHORS OR COPYRIGHT HOLDERS BE LIABLE FOR ANY CLAIM, DAMAGES OR OTHER
%% LIABILITY, WHETHER IN AN ACTION OF CONTRACT, TORT OR OTHERWISE, ARISING FROM,
%% OUT OF OR IN CONNECTION WITH THE SOFTWARE OR THE USE OR OTHER DEALINGS IN THE
%% SOFTWARE.

% The font could be set to Windows-specific Calibri by using the 'calibri' option
\documentclass[]{mcdowellcv}

% For mathematical symbols
\usepackage{amsmath}
\usepackage{hyperref}

% Set applicant's personal data for header
\name{Christian Rebischke}
\address{Marie-Hedwig-Straße 13 \linebreak Apartment 321 \linebreak Clausthal-Zellerfeld 38678 \linebreak Germany}
\contacts{+49 151 6190 2666 \linebreak chris@shibumi.dev \linebreak \url{https://shibumi.dev}}

\begin{document}

% Print the header
\makeheader

\begin{cvsection}{Education}
\begin{cvsubsection}{M.Sc.\@ in Computer Science}{Clausthal University of Technology}{Oct 2018 --- Apr 2021}
\bigskip
\begin{itemize}
\item \textbf{Key Coursework}: Advanced computer networks, cloud computing, high performance computing, high performance computing with C++, network security
\end{itemize}
\end{cvsubsection}
\begin{cvsubsection}{B.Sc.\@ in Computer Science}{Clausthal University of Technology}{Oct 2013 --- May 2019}
\bigskip
\begin{itemize}
\item \textbf{Key Coursework}: Algorithms and data structures, computer architecture, computer networks, databases, operating systems and distributed systems, software engineering. GPA:\@ 2.6
\end{itemize}
\end{cvsubsection}

\begin{cvsection}{Work Experience}
\begin{cvsubsection}{Google Summer of Code}{in-toto, CNCF}{June 2020 --- Sep 2020}
\begin{itemize}
\item Ported \textit{runlib} functionality from the \textit{in-toto} reference implementation in Python to the \textit{in-toto} implementation in Go by expanding the library with code for signing and generating \textit{in-toto} link metadata.
\item Personal highlight: Found a bug in the crypto/rsa Go library and submit a PR to Gerrit.
\end{itemize}
\end{cvsubsection}
\begin{cvsubsection}{Work Student}{Avency, Remote}{Apr 2020 --- Now}
\begin{itemize}
\item Developed a package manager in Go for deploying static binaries on NextGen Firewalls.
\end{itemize}
\end{cvsubsection}
\begin{cvsubsection}{Work Student}{Clausthal University of Technology}{Apr 2014 --- Apr 2020}
\bigskip
\begin{itemize}
\item Automated the TLS keypair deployment to a central firewall for inbound TLS inspection by writing a middleware in Python that pulls TLS keypairs and pushes them via REST-API to the firewall.
\item Achieved a relation of LDAP users and IP addresses for writing user/IP specific firewall rules via implementing
a REST API as middleware between a proprietary service, Freeradius and OpenVPN
\item Reduced toil of writing 2--25 mails daily via writing a software in Pythont that fetches IPS firewall alerts via REST API and mails them
to responsible system administrators in institutes.
\item Showed ownership by maintaining a Proxmox VE cluster consisting of 25 physical nodes.
\item Wrote a software that helps finding unused or orphaned artifacts in the firewall and therefore reduced the manual task of searching for them
\item Evaluated Kubernetes for increasing reliability and introducing micro segmentation via namespace segregation
\end{itemize}
\end{cvsubsection}
\end{cvsection}

\begin{cvsection}{Overview}
\begin{cvsubsection}{}{}{}
\begin{itemize}
\item \textbf{Natural Languages} German, English
\item \textbf{Programming Languages} Golang, Python, Bash, C++
\item \textbf{Interests} Site-Reliability Engineering, Devops, Linux, Security
\end{itemize}
\end{cvsubsection}
\end{cvsection}

\end{cvsection}
\begin{cvsection}{Selected Projects}
\begin{cvsubsection}{}{}{}
\begin{itemize}
\item \textbf{Arch Linux} Working for Arch Linux as package maintainer and security team member since 2015
\item \textbf{Arch Linux Boxes} Automated monthly Vagrant and qcow2 image builds with Ansible and Hashicorp Packer. Reduced the toil of 1 hour per month to manually check for the monthly needed fresh Arch Linux ISO image via writing a Python script for monitoring the ISO image releases.
\item \textbf{CIFS-exporter} A SMB/CIFS Prometheus Exporter, that parses \textit{/proc/fs/cifs/Stats} and exposes them via an HTTP server for Prometheus.
\item \textbf{mnemonic} Diceware alike memorizeable password generator written in Go.
\item \textbf{ansible-hcloud-inventory} A dynamic \textit{Ansible} inventory for the \textit{Hetzner Cloud}.
\end{itemize}
\end{cvsubsection}
\end{cvsection}

\end{document}
