\documentclass[11pt,a4paper,nolmodern]{moderncv}

\title{DevOps Engineer}

\usepackage{shibumi}

%% start of file `template.tex'.
%% Copyright 2006-2010 Xavier Danaux (xdanaux@gmail.com).
%
% This work may be distributed and/or modified under the
% conditions of the LaTeX Project Public License version 1.3c,
% available at http://www.latex-project.org/lppl/.

% Version: 20110122-4


\usepackage[onehalfspacing]{setspace}
\usepackage{fontspec}
\usepackage[english]{babel}
\linespread{1.18}
% for some reason, lines take up a lot of space in itemize in English...
\newenvironment{tightitemize}
   {\begin{itemize}
   \setlength{\parskip}{0pt}}
   {\end{itemize}}


% personal data
\extrainfo{%
\xing~\httplink{xing.com/profile/shibumi}\\%
\faGithub~\httplink{github.com/shibumi}%
% \faCar~Führerschein Klasse B} % optional, remove the line if not wanted

\myquote{DevOps is all about empathy}{Rob Nelson}


%\nopagenumbers{}                             % uncomment to suppress automatic page numbering for CVs longer than one page
\begin{document}
\setmainfont{TeX Gyre Pagella}
%\setsansfont{Myriad Pro}

\hyphenpenalty=10000
%\maketitle



\maketitle

\section{Fähigkeiten}

\subsection{Entwicklung}
\cvcomputer{Sprachen}{C, C++, Python, Golang, Shell/Bash, x86 Assembler, Java, LaTeX, HTML, Markdown}{}{}
\cvcomputer{Werkzeuge}{GitLab, GitHub, Travis CI, Git, SVN, Jetbrains IDEs, Vim, Hashicorp Vagrant, SonarQube, Gradle, UML}
           {Methoden}{Scrum, V-Modell XT, Agile, Test-Driven-Development, User-Experience (UX), Design-Patterns, Kanban, Requirements-Engineering}

\subsection{System- und Netzwerk-Administration}
\cvcomputer{Web}{Apache, Lighttpd, Nginx}
           {Monitoring}{NSCA, NRPE, SNMP, Icinga, Nagios, Centreon, Prometheus, Grafana}
\cvcomputer{Mail}{Postfix}
           {Backup}{Borg, Rsync}
\cvcomputer{\kern-3ex Automatisierung}{Ansible, Hashicorp Packer, NAPALM}
           {DNS}{DNSMasq}
\cvcomputer{\kern-3ex Betriebssysteme}{GNU/Linux (Debian, Ubuntu, CentOS, Archlinux)}
           {Datenbanken}{MySQL, MariaDB, SQLite}
\cvcomputer{\kern-1em Virtualisierung}{VMWare, QEMU/KVM, libvirt, Virtualbox}
           {Paketbau}{pkg.tar.xz, RPM, Deb}
\cvcomputer{Container}{Docker, Systemd-Nspawn}
           {Netzwerk}{EVPN-BGP, VXLAN, IP Fabrics, NextGen Firewalls, TCP/IP, UDP/IP, ARP, HTTP}
\cvcomputer{Init-Systeme}{SysVinit, Systemd}
           {Storage}{NFSv4, CIFS}
\cvcomputer{Firewalls}{Forcepoint NGFW, Iptables, Firewalld}
           {Linux}{PAM, CGroups, Kernel, Polkit, DBus}
\cvcomputer{VPN}{OpenVPN, Wireguard}
           {Infrastruktur}{LDAP, DNS}

\newpage
\subsection{IT-Sicherheit}
\cvcomputer{Forensik}{Photorec}
           {Malware Analyse}{Yara, GDB, strace, radare2, IDA Pro}
\cvcomputer{Offensive}{Blackbox/Whitebox-Tests, Exploits, Fuzzing, Informationsgewinnung}
           {Defensive}{Linux-Hardening (CGroups, Namespaces, Selinux),
           Verschlüsselung (dm-crypt, LUKS, OpenSSL, OpenSSH), Honeypots
           (Kippo, cowrie), Port-Knocking, Anonymisierung, Obfuscation}
\newpage

\section{Erfahrung}

\tlcventry{2016}{0}{Hilfswissenschaftler}{\href{https://rz.tu-clausthal.de}{Rechenzentrum, TU Clausthal}}{Clausthal-Zellerfeld, Niedersachsen}{}{Systems und Network Engineer
\begin{itemize}
  \item Systems Engineer:
    \begin{itemize}
      \item Monitoring mit Centreon;
      \item NFSv4 mit Kerberos Proof of Concept;
      \item Systemadministration (CentOS, Ubuntu);
      \item Configuration Management (Ansible, OPSI);
      \item Package Management (RPM, DEB);
      \item Migration eines Libvirt Clusters zu Proxmox;
      \item Automatisierung von VMware Proof of Concept;
      \item Betrieb einer OpenVAS Instanz;
      \item Security Engineering.
    \end{itemize}
  \item Network Engineer:
    \begin{itemize}
      \item Entwicklung diverser Automatisierungswerkzeuge für die Forcepoint NGFW mit Python;
      \item EVPN-BGP Proof of Concept und dessen Automatisierung mit Ansible;
      \item Automatisierung von Switches mit Ansible-NAPALM;
      \item Entwicklung einer Automatisierungslösung für Deep Packet Inspection für die Forcepoint NGFW mit Python.
    \end{itemize}
\end{itemize}}

\tlcventry{2016}{2017}{Hilfswissenschaftler}{\href{https://www.in.tu-clausthal.de}{Institut für Informatik, TU Clausthal}}{Clausthal-Zellerfeld, Niedersachsen}{}{Software Engineer
\begin{tightitemize}
 \item Versionskontrolle mit SVN;
 \item Software-Architektur;
 \item Software-Entwickung in Java;
 \item Build-Management mit Gradle;
 \item FMI (Functional Mockup Interface);
 \item Matlab Simulink Compiler;
 \item Continuous Code Quality mit Sonarqube.
\end{tightitemize}}

\tlcventry{2014}{2017}{Hilfswissenschaftler}{\href{https://math.tu-clausthal.de}{Institut für Mathematik, TU Clausthal}}{Clausthal-Zellerfeld, Niedersachsen}{}{System Administrator
\begin{tightitemize}
  \item Systemadministration (Windows, GNU/Linux);
  \item Identity and Access Management (LDAP, PAM, Kerberos);
  \item Monitoring (NRPE, NSCA, SNMP);
  \item Package Management (RPM, DEB);
  \item Bash-Scripting;
  \item Printserver (CUPS);
  \item Webserver (Apache);
  \item Network-Storage (NFSv4, Firefox-Sync-Server);
  \item Datenbanken (MySQL);
  \item Configuration Management (OPSI).
\end{tightitemize}}

\tlcventry{2017}{0}{Trusted User}{\href{https://archlinux.org}{Arch Linux}}{Internet}{}{Package Maintainer
\begin{tightitemize}
  \item Pflege von Linux Paketen für die Distribution \href{https://archlinux.org}{Arch Linux};
  \item Härten von Paketen;
  \item Einpflegen von Patches;
  \item Entwicklung einer Umgebung für automatisierte Vagrant Image Releases: \href{https://github.com/archlinux/arch-boxes}{https://github.com/archlinux/arch-boxes}.
\end{tightitemize}}

\tlcventry{2015}{0}{Security Team Mitglied}{\href{https://archlinux.org}{Arch Linux}}{Internet}{}{Security Engineer
\begin{tightitemize}
  \item CVE-Monitoring;
  \item Härten von Paketen;
  \item Schreiben von Security Patches.
\end{tightitemize}}

\section{Ausbildung}

\tlcventry{2013}{0}{Studium der Informatik (Bachelor)}{TU Clausthal, Niedersachsen}{}{}{}

\tlcventry{2004}{2013}{Abitur}{}{Robert-Koch-Gymnasium, Clausthal-Zellerfeld, Niedersachsen}{}{}{}

\section{Ehrenamt}

\tlcventry{2006}{2013}{Freiwillige Feuerwehr}{Hahnenklee-Bockswiese, Niedersachsen}{}{}{%
\begin{tightitemize}
   \item Erreichen des Dienstgrads Hauptfeuerwehrmann;
   \item Bestehen des Lehrgangs Atemschutzgeräteträger;
   \item Bestehen des Lehrgangs Sprechfunker;
   \item Bestehen des Grundlehrgangs;
   \item Ehrenamtliche Tätigkeit als Atemschutzgerätewart;
\end{tightitemize}}

\section{Veröffentlichungen}

\tldatecventry{2018}{Seminararbeit über OpenStack}{\href{https://github.com/shibumi/openstack-project-paper/blob/master/seminararbeit.pdf}{https://github.com/shibumi/openstack-project-paper/blob/master/seminararbeit.pdf}}{}{}{}{}

\tldatecventry{2017}{Seminararbeit über Amazon AWS}{\href{https://github.com/shibumi/aws-ec2-project-paper/releases/download/final-release/essay.pdf}{https://github.com/shibumi/aws-ec2-project-paper/releases/download/final-release/essay.pdf}}{}{}{
\begin{tightitemize}
  \item Automatisierung des Builds der LaTeX basierenden Seminararbeit mit Travis CI
\end{tightitemize}}

\tldatecventry{2015}{Network Security Vorlesung Projektpapier: Tor}{\href{https://github.com/shibumi/Tor-project-paper/blob/master/tor.pdf}{https://github.com/shibumi/Tor-project-paper/blob/master/tor.pdf}}{}{}{}{}

\section{Fremdsprachen}

\cvlanguage{Englisch}{Fließend}{Tägliche Nutzung}
\cvlanguage{Russisch}{Grundkenntnisse}{4 Jahre Schule}
\cvlanguage{Japanisch}{Grundkenntnisse}{Freizeitbeschäftigung}
\cvlanguage{Chinesisch}{Grundkenntnisse}{Freizeitbeschäftigung}

\end{document}
