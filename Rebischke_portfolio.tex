%% The MIT License (MIT)
%%
%% Copyright (c) 2015 Daniil Belyakov
%%
%% Permission is hereby granted, free of charge, to any person obtaining a copy
%% of this software and associated documentation files (the "Software"), to deal
%% in the Software without restriction, including without limitation the rights
%% to use, copy, modify, merge, publish, distribute, sublicense, and/or sell
%% copies of the Software, and to permit persons to whom the Software is
%% furnished to do so, subject to the following conditions:
%%
%% The above copyright notice and this permission notice shall be included in all
%% copies or substantial portions of the Software.
%%
%% THE SOFTWARE IS PROVIDED "AS IS", WITHOUT WARRANTY OF ANY KIND, EXPRESS OR
%% IMPLIED, INCLUDING BUT NOT LIMITED TO THE WARRANTIES OF MERCHANTABILITY,
%% FITNESS FOR A PARTICULAR PURPOSE AND NONINFRINGEMENT. IN NO EVENT SHALL THE
%% AUTHORS OR COPYRIGHT HOLDERS BE LIABLE FOR ANY CLAIM, DAMAGES OR OTHER
%% LIABILITY, WHETHER IN AN ACTION OF CONTRACT, TORT OR OTHERWISE, ARISING FROM,
%% OUT OF OR IN CONNECTION WITH THE SOFTWARE OR THE USE OR OTHER DEALINGS IN THE
%% SOFTWARE.

% The font could be set to Windows-specific Calibri by using the 'calibri' option
\documentclass[]{mcdowellcv}

% For mathematical symbols
\usepackage{amsmath}
\usepackage{hyperref}

% Set applicant's personal data for header
\name{Christian Rebischke}
\address{Marie-Hedwig-Straße 13 \linebreak Apartment 321 \linebreak Clausthal-Zellerfeld 38678}
\contacts{+49 151 6190 2666 \linebreak chris@shibumi.dev \linebreak \url{https://shibumi.dev}}

\begin{document}

% Print the header
\makeheader%

\begin{cvsection}{Education}
\begin{cvsubsection}{M.Sc.\@ in Computer Science}{Clausthal University of Technology}{Oct 2018 --- Apr 2021}
\bigskip
\begin{itemize}
\item \textbf{Key Coursework}: Advanced computer networks, cloud computing, high performance computing, high performance computing with C++, network security
\end{itemize}
\end{cvsubsection}
\begin{cvsubsection}{B.Sc.\@ in Computer Science}{Clausthal University of Technology}{Oct 2013 --- May 2019}
\bigskip
\begin{itemize}
\item \textbf{Key Coursework}: Algorithms and data structures, computer architecture, computer networks, databases, operating systems and distributed systems, software engineering. GPA:\@ 2.6
\end{itemize}
\end{cvsubsection}

% Print the content
\begin{cvsection}{Employment}
\begin{cvsubsection}{Work Student}{Avency, Remote}{Apr 2020 --- Now}
\begin{itemize}
\item Showed Ownership in maintaining an on-premise Kubernetes Cluster
\item Developed a package manager for deploying static binaries in Golang for NextGen Firewalls.
\end{itemize}
\end{cvsubsection}
\begin{cvsubsection}{Work Student}{TU Clausthal, Datacenter (Network Department)}{Apr 2017 --- Apr 2020}
\bigskip
\begin{itemize}
\item Wrote a software that helps finding unused or orphaned artifacts in firewalls..
\item Built a proof of concept for deploying Virtual Tunnel End Points (VTEPs) with Ansible on Linux machines for EVPN BGP/VXLAN.
\item Implemented an automated system in Python for fetching IPS firewall alerts via REST API and mailing them to responsible system administrators. This reduced the toil of writing 5--25 mails daily.
\item Reduced MTTR from one work day to one hour by automating a Freeradius/Radsecproxy based AAA infrastructure with Ansible.
\item Designed and implemented a command line tool in Python for deploying TLS certificates and private keys on a central firewall for inbound TLS inspection. The production environment had 4000 students and 1000 employees.
\item Evaluated Kubernetes for increasing reliability and introducing micro segmentation via namespace segregation.
\item Set up a distributed monitoring system with the help of Traefik, Prometheus and Grafana for monitoring Service Level Indicators (SLIs) for different institutions within the university campus.
\item Gave a talk about Freeradius and Radsecproxy deployment via Ansible on the DFN-BT (annual German research network meetup).
\item Achieved a relation of LDAP users and IP addresses for writing user/IP specific firewall rules via implementing a REST API as middleware between a proprietary service, Freeradius and OpenVPN.\@
\item Additional key technologies being used: NSCA, NRPE, SNMP, Nginx, Apache, NAPALM, NFSv4 over Kerberos, Elasticsearch, Logstash, Kibana, Ansible, Python-Flask, Docker, Git.
\end{itemize}
\end{cvsubsection}

\begin{cvsubsection}{Work Student}{TU Clausthal, Datacenter (Systems Department)}{Apr 2016 --- Apr 2017}
\bigskip
\begin{itemize}
\item Worked on a proof of concept for a distributed filesystem based on NFSv4 and Kerberos.
\item Evaluated possibilities for the usage of OPSI for software deployment on Windows clients.
\item Improved system security and reliability by setting up an OpenVAS vulnerability scanner for internal servers and housing.
\item Showed ownership by maintaining a Proxmox VE cluster consisting of 25 physical nodes.
\item Contributed to monitoring via setting up a status website based on Cachet and adding checks in the monitoring plattform Centreon with NRPE and SNMP requests.
\end{itemize}
\end{cvsubsection}

\begin{cvsubsection}{Work Student}{TU Clausthal Inst.\@ of Software Systems Engineering}{Oct 2016 --- Sep 2017}
\bigskip
\begin{itemize}
\item Built a tool chain for exporting Matlab Simulink models into the Functional Mockup Unit (FMU) format.
\item Developed components for a model transformation tool suite in the project \emph{Spectral Analsysis of Software Architecture}
\item Enhanced code quality by establishing the Continuous Code Quality tool Sonarqube.
\item Key technologies being used: Java, Gradle, Matlab, SVN
\end{itemize}
\end{cvsubsection}

\begin{cvsubsection}{Work Student}{TU Clausthal Inst.\@ of Mathematics}{Apr 2014 --- Sep 2017}
\bigskip
\begin{itemize}
\item Increased system reliability by monitoring via the Nagios fork Centreon.
\item Build software packages for Ubuntu (deb) and CentOS (rpm).
\item Has been the system administrator for Linux and Windows machines and gave first level support.
\item Technologies being used: Bash, NFSv4 with Kerberos, Apache, CUPS, MySQL\@.
\end{itemize}
\end{cvsubsection}
\end{cvsection}

\begin{cvsection}{Overview}
\begin{cvsubsection}{}{}{}
\begin{itemize}
\item \textbf{Natural Languages} German, English
\item \textbf{Programming Languages} Go, Python, Bash
\item \textbf{Interests} Site Reliability Engineering, Devops, Linux Security
\item \textbf{Favourite Technologies} Hashicorp products (Terraform, Packer), Kubernetes, Ansible, Github Actions / Gitlab CI
\end{itemize}
\end{cvsubsection}
\end{cvsection}

\end{cvsection}
\begin{cvsection}{Selected Projects}
\begin{cvsubsection}{}{}{}
\begin{itemize}
\item \textbf{Arch Linux} Working on Arch Linux as package maintainer for cloud related packages (Hashicorp Vault, Packer, Kubermatic Kubeone, Kubectl, Fluxctl, etc) and security team member since 2015.
\item \textbf{in-toto} Framework to secure the integrity of software supply chains.
\item \textbf{CIFS-exporter} A SMB/CIFS Prometheus Exporter, that parses \textit{/proc/fs/cifs/Stats} and exposes them via an HTTP server for Prometheus.
\item \textbf{mnemonic} Diceware alike memorizeable password generator written in Go.
\item \textbf{ansible-hcloud-inventory} A dynamic \textit{Ansible} inventory for the \textit{Hetzner Cloud}.
\item \textbf{Hikari} A unique ZSH configuration with some special key combinations
\end{itemize}
\end{cvsubsection}
\end{cvsection}

\end{document}
